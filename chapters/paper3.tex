% Previous work into shearing bijels have demonstrated how particles prefer to move in the direction of shear and if strong enough, even detach from the interface. \cite{bonaccorso_shear_2020} It has also been shown how at curved interfaces, ellipsoidal particles  

While the microstructure and synthesis techniques for bijels have been explored in earnest, the rheology of bijels
has only recently seen an uptick in interest. Some interesting phenomena for bijels and shearing include the 
presence of monogels, made by remixing the phase separated liquid domains of the bijel while the electrostatic
interactions between particles maintains the structure of the particle monolayer. \textcolor{blue}{https://journals.aps.org/prl/abstract/10.1103/PhysRevLett.103.255502}
This behavior is system specific, suggesting the fluid system plays a role in this behavior as well. \textcolor{blue}{https://doi.org/10.1002/adfm.201002562} 
\cite{bai_dynamics_2015} Investigations have identified that confined bijels under shear undergo elongation of
the domains in the direction of shear succeeded by particle detachment from the interface and eventual failure of the bijel. \cite{bonaccorso_shear_2020}
More recently, bijels have been identified to be 2D glasses percolating in 3D space, characterized through comparing the complex rheology of bijels against
colloidal gels made from silica particles with electrostatic interactions. \cite{ching_bijel_2022} 

In many of the manufacturing techniques outlined for continuous bijel production such as STrIPS, the rheological properties are essential in ensuring that 
the casting mixture remains processable. \cite{haase_continuous_2015,haase_situ_2016} While particle stabilized emulsions have been shown to be shear thinning, 
our investigation into magnetically stimuli responsive bijels will modify the behavior of these systems due to the application of magnetic fields. Past work with 
ferrofluids and magnetic emulsions have showed how the rheological properties vary drastically as a function of the field strength due to the orientation of the 
particles. \textcolor{blue}{https://doi.org/10.1016/j.colsurfa.2012.06.026} The orientation of the particle to shear as well as the nematic order of particles 
can cause shear banding and a smaller effective viscosity owing to a smaller cross sectional area being presented to the direction of shear. 

In this chapter, we probe the dynamics of bijels under constant and complex shear to understand how the application of magnetic fields onto bijels stabilized with
anisotropic particles. We define a shear capillary number $Ca_s = \frac{\eta_{f} \dot{\gamma} L_{1}}{\sigma}$ where $\dot{\gamma} = \frac{2u_{LE}}{L_x}$ is the
strain rate and $L_1$ is the average domain size. \cite{frijters_effects_2012, yang_capillary_2022} In the literature, $Ca_s$ has been between between 0.04 
and 0.16. However, the box size in these simulations were smaller, meaning that these capillary number ranges would exceed the 
maximum mach number the model allows if this same range were used $(Ma \leq 0.03)$. To accommodate this limitation the 
largest capillary number used will be $Ca_s = 10^-5$ which correspond to a maximum of $Ma \approx 0.002$. $Ca_s = 10^-6,  
10^-7$ will also be used to demonstrate how the bijels respond to shear of varying strengths. As all particles used 
have only hard-sphere type interactions, it is expected that the behaviour seen should mimic 2D colloidal glasses 
percolating in 3D space, akin to what Ching and Mohraz saw, with an additional dependence on -the direction of shear. 
This would predict the discovery of particle monolayer dependent elasticity and yield stress along with shear thinning 
behaviour as a function of strain rate. 

To verify these predictions, bijel microstructure will be defined using four processing histories; The first is of a 
bijel simulated under a $\Bar{B} = 1$ magnetic field strength, the second is a bijel stimulated under no field, 
followed by the application of a $\Bar{B} = 1$ field after jamming, the third is a bijel simulated under no field, 
while the final microstructure is a bijel simulated under $\Bar{B} = 1$ magnetic field, followed by switching the 
field off after jamming. This gives insight into the impact of processing history on the shear properties of a bijel, 
in addition to the microstructural and colloidal insights gained. Based on the results in Bonaccorso et al., there 
should be shear driven and shear rate dependent elongation of the domains in the direction parallel to the applied 
shear which in this system will be seen as a reduction in $L_{\perp}$ and an increase in $L_{\parallel}$. 
\cite{bonaccorso_shear_2020} The microstructure anisotropy will also be a factor in the viscosity results, as 
larger domains are more permeable than smaller ones, meaning that bijels where $L_{\perp} > L_{\parallel}$ should 
see less of the domain elongation effects shown in Bonaccorso et al. as the permeability of the bijels rises with 
larger domain size. \cite{bonaccorso_shear_2020}

It is also expected that the effective viscosity will be different between the four microstructures dependent upon the 
degree of nematic ordering and microstructural anisotropy of the bijel. Nematic ordering of the particles will be mimic 
shear banding in colloidal suspensions, resulting in a lowered effective viscosity compared to bijels without nematically 
ordered particles. \cite{xu_relation_2013, vermant_flow-induced_2005} Tracking of the proportion of particles on the 
interface will also yield insight into how the packing of the particles affects the rate at which particles will get 
ejected from the interface. Systems with a larger $\eta_{eff}$ are predicted to have the largest domains, largest 
difference between initial and final particle order and lowest number of particles left on the interface once steady 
state has been established. 

\textcolor{blue}{https://doi.org/10.1073/pnas.0812519106}

\begin{itemize}
    \item Shear banding analyzed through looking at change in particle velocity in the x direction
    \item Shear banding analyzed through looking at change in $Q_6$ in the x direction
    \item 
\end{itemize}

\section{Results}\label{sec:results_p3}
\subsection{Microstructure change}

\subsection{Particle properties}

\subsection{Shear banding}

\subsection{Viscosity measurement}