% Previous work into shearing bijels have demonstrated how particles prefer to move in the direction of shear and if strong enough, even detach from the interface. \cite{bonaccorso_shear_2020} It has also been shown how at curved interfaces, ellipsoidal particles  

In many of the manufacturing techniques outlined for continuous bijel production such as STrIPS, the rheological properties are essential in ensuring that the casting mixture remains processable. However, past work with ferrofluids and magnetic emulsions have showed how the rheological properties vary drastically as a function of the field strength. Bijels themselves have complex rheological properties stemming from the removal of particles from the interface if too much shear is applied. Therefore, avenues of exploration here would be how the rheological properties of bijels change under the influence of fields in response to an applied shear. One of the questions that will be investigated in particular will be how shear changes the orientation of the particles on the interface, the interfacial coverage of particles and if there is anisotropy in the viscosity during the application of a field owing to the orientation of particles towards the field.

First, a shear capillary number is defined as $Ca_s = \frac{\eta_{f} \dot{\gamma} L_{1}}{\sigma}$ where $\dot{\gamma} = \frac{u_{LE}}{L_x}$ is the strain rate and $L_1$ is the average domain size. \cite{frijters_effects_2012, yang_capillary_2022} In the literature, $Ca_s$ has been between between 0.04 and 0.16. However, the box size in these simulations were smaller, meaning that these capillary number ranges would exceed the maximum mach number the model allows if this same range were used $(Ma \leq 0.03)$. To accommodate this limitation the largest capillary number used will be $Ca_s = 10^-5$ which correspond to a maximum of $Ma \approx 0.002$. $Ca_s = 10^-6,  10^-7$ will also be used to demonstrate how the bijels respond to shear of varying strengths. As all particles used have only hard-sphere type interactions, it is expected that the behaviour seen should mimic 2D colloidal glasses percolating in 3D space, akin to what Ching and Mohraz saw, with an additional dependence on -the direction of shear. This would predict the discovery of particle monolayer dependent elasticity and yield stress along with shear thinning behaviour as a function of strain rate. 

To verify these predictions, bijel microstructure will be defined using four processing histories; The first is of a bijel simulated under a $\Bar{B} = 1$ magnetic field strength, the second is a bijel stimulated under no field, followed by the application of a $\Bar{B} = 1$ field after jamming, the third is a bijel simulated under no field, while the final microstructure is a bijel simulated under $\Bar{B} = 1$ magnetic field, followed by switching the field off after jamming. This gives insight into the impact of processing history on the shear properties of a bijel, in addition to the microstructural and colloidal insights gained. Based on the results in Bonaccorso et al., there should be shear driven and shear rate dependent elongation of the domains in the direction parallel to the applied shear which in this system will be seen as a reduction in $L_{\perp}$ and an increase in $L_{\parallel}$. \cite{bonaccorso_shear_2020} The microstructure anisotropy will also be a factor in the viscosity results, as larger domains are more permeable than smaller ones, meaning that bijels where $L_{\perp} > L_{\parallel}$ should see less of the domain elongation effects shown in Bonaccorso et al. as the permeability of the bijels rises with larger domain size. \cite{bonaccorso_shear_2020}

It is also expected that the effective viscosity will be different between the four microstructures dependent upon the degree of nematic ordering and microstructural anisotropy of the bijel. Nematic ordering of the particles will be mimic shear banding in colloidal suspensions, resulting in a lowered effective viscosity compared to bijels without nematically ordered particles. \cite{xu_relation_2013, vermant_flow-induced_2005} Tracking of the proportion of particles on the interface will also yield insight into how the packing of the particles affects the rate at which particles will get ejected from the interface. Systems with a larger $\eta_{eff}$ are predicted to have the largest domains, largest difference between initial and final particle order and lowest number of particles left on the interface once steady state has been established. 
