% 
% This is the Title and Approval Page file (dissertation-tap.tex) for
% a dissertation.
%
% The order of the commands below is very important.
% You may choose to add or eliminate a \prefacesection 
% in the front material but the order should remain 
% the same especially \maketocloflot followed by 
% \prefacesectiontoc{Abstract}

% Title and author are also used for PDF file properties
% No special character or commands can be used for the PDF definition; 
% use the [options] paramater to specify a different title or author 
% to remove special characters or commands like \\ for example.
\title[First Line of Title Second Line of Title]{Lattice Boltzmann simulations of magnetically responsive\\
ellipsoidal particle stabilized emulsions}
\author{Nikhil Karthikeyan}
\type{dissertation}
\degree{Doctor of Philosophy}
\majorfieldtrue\majorfield{Materials Science and Engineering}
\degreedate{Spring 2025}
% Optional PDF properties
\keywords{Keyord,Keyword,Keyword}
\subject{Subject}

\maketitlepage % Generates Title Page

\begin{approvalpage}
\chair{Joshua Zide, PhD}{Chair of the Department of Materials Science and Engineering}
\dean{Pamela M. Norris, PhD}{Dean of the College of Engineering}
\end{approvalpage}

\begin{signedpage} % Up to 4 signatures
\profmember{Ulf D. Schiller, PhD}
\member{Eric Furst, PhD}
\member{Arthi Jayaraman, PhD}
\member{Darrin Pochan, PhD}
\end{signedpage}

% For additional signatures beyond 4, uncomment and use
% \begin{signedpagecont}
% \member{Xxxx Xxxx, Highest Degree}
% \member{Xxxx Xxxx, Highest Degree}
% \end{signedpagecont}

\begin{front} % Starts front material (Roman style page numbers)

\prefacesection{Acknowledgements}
%\input{acknowl} % This file (acknowl.tex) contains the text
                % for the acknowledgments or type text here.

In the development of this dissertation, I extend my deepest gratitude to Dr Ulf Schiller whose support, motivation and 
opportunities have been indispensible during my PhD. I believe his methodical problem solving strategies, depth and 
breadth of knowledge have been invaluable sources of inspiration for me as well as 
motivating myself to strive towards those ideals. I would also like to thank members of my committee, 
from the University of Delaware and Clemson University for their mentorship 
and motivation that pushed me to increase my breadth of knowledge and to think outside the box.

Listing everyone I am grateful for who have provided me non-academic support would be too long for this page 
although I shall try to summarize this. I am grateful to my family, with whom I may not be the best at 
discussing my future plans, but they have been an indispensible source of support during my time away from home. 
I am thankful to my lab members especially for adding brief moments of levity during my PhD and giving me the 
opportunity to be a part of their own journey in graduate school. Thank you to everyone who has been with me 
through thick and thin, and to many more celebrations of our achievements. 

% Table of Contents is always created, but you
% may set \tablespagefalse and \figurespagefalse 
% if you don't want these generated automatically
% (i.e. List of Tables and List of Figures).
% These are set to true by default (i.e. \tablespagetrue,
% \figurespagetrue).

% Uncomment if you do not want a List of Figures.
%\figurespagefalse

% Uncomment if you do not want a List of Tables.
%\tablespagefalse 

\maketocloflot

\prefacesectiontoc{Abstract}
%\input{abstract} % This file (abstract.tex) contains the text
                 % for an abstract or type text here.

Porous materials have found great importance in applications such as water filtration, catalyst supports, battery 
electrodes and bio engineered materials. To accommodate a greater variety of pore sizes and continuous fabrication 
techniques with minimal waste, bottom up synthesis techniques have been gaining in popularity. One such technique is
emulsion templating of a specific microstructure known as a bicontinuous interfacially jammed emulsion gels(bijels). 

However, casting mixture independent microstructure control is an open issue, especially applicable in continuous 
bijel synthesis strategies such as Solvent Transfer Induced Phase separation and as in-situ control mechanisms to
control the microstructure of a bijel to control properties such as separation efficiency. Stimuli response has been identified as a 
technique to enable this. This study focuses on utilizing stimuli response of bijels stabilized with ellipsoidal microparticles, 
and the resulting microstructure and rheological implications this induces. 

We first identify that upon application of a field to a bijel stabilized with ellipsoidal particles during synthesis, 
microstructure change is observed, not present when analyzing bijels stabilized with spherical particles. Upon 
application of the field, particles reorient to the direction of the field, causing directional jamming that creates
microstructure anisotropy.

We then identify that bijels stabilized with ellipsoidal particles have tunable microstructures after jamming. 
This process is dependent upon the initial particle order present in the bijel, as well as the difference 
between the initial and final particle order. We also show that the structural response observed is state 
dependent, meaning that the time of magnetic field application affects the microstructure change observed.

Finally, we characterize that bijels retain their shear thinning behavior when analyzing the rheological response of bijels
under magnetic fields. We identify that the particle ordering plays a role in the evolution of the microstructure under shear
and the rheological response. 

\end{front}

