% 
% This is the Title and Approval Page file (dissertation-tap.tex) for
% a dissertation.
%
% The order of the commands below is very important.
% You may choose to add or eliminate a \prefacesection 
% in the front material but the order should remain 
% the same especially \maketocloflot followed by 
% \prefacesectiontoc{Abstract}

% Title and author are also used for PDF file properties
% No special character or commands can be used for the PDF definition; 
% use the [options] paramater to specify a different title or author 
% to remove special characters or commands like \\ for example.
\title[First Line of Title Second Line of Title]{Emulsions Stabilized by Magnetic Ellipsoidal Particles: A Lattice Boltzmann Study}
\author{Nikhil Karthikeyan}
\type{dissertation}
\degree{Doctor of Philosophy}
\majorfieldtrue\majorfield{Materials Science and Engineering}
\degreedate{Spring 2025}
% Optional PDF properties
\keywords{Keyord,Keyword,Keyword}
\subject{Subject}

\maketitlepage % Generates Title Page

\begin{approvalpage}
\chair{Joshua Zide, PhD}{Chair of the Department of Materials Science and Engineering}
\dean{Pamela M. Norris, PhD}{Dean of the College of Engineering}
\end{approvalpage}

\begin{signedpage} % Up to 4 signatures
\profmember{Ulf D. Schiller, PhD}
\member{Eric M. Furst, PhD}
\member{Arthi Jayaraman, PhD}
\member{Darrin Pochan, PhD}
\end{signedpage}

% For additional signatures beyond 4, uncomment and use
% \begin{signedpagecont}
% \member{Xxxx Xxxx, Highest Degree}
% \member{Xxxx Xxxx, Highest Degree}
% \end{signedpagecont}

\begin{front} % Starts front material (Roman style page numbers)

\prefacesection{Acknowledgements}
%\input{acknowl} % This file (acknowl.tex) contains the text
                % for the acknowledgments or type text here.

In the development of this dissertation, I extend my deepest gratitude to Dr Ulf Schiller whose support, motivation and 
opportunities have been indispensible during my PhD. I believe his methodical problem solving strategies, depth and 
breadth of knowledge have been invaluable sources of inspiration for me as well as 
motivation to strive towards those ideals. I would also like to thank members of my committee, 
from the University of Delaware and Clemson University for their mentorship 
and motivation that pushed me to increase my breadth of knowledge and to think outside the box.

Listing everyone I am grateful for who have provided me non-academic support would be too long for this page. 
I am grateful to my family, with whom I may not be the best at 
discussing my future plans, but they have been an indispensible source of support during my time away from home. 
I am thankful to my lab members especially for adding brief moments of levity during my PhD and giving me the 
opportunity to be a part of their own journey in graduate school. Thank you to everyone who has been with me 
through thick and thin, and to many more celebrations of our achievements. 

% Table of Contents is always created, but you
% may set \tablespagefalse and \figurespagefalse 
% if you don't want these generated automatically
% (i.e. List of Tables and List of Figures).
% These are set to true by default (i.e. \tablespagetrue,
% \figurespagetrue).

% Uncomment if you do not want a List of Figures.
%\figurespagefalse

% Uncomment if you do not want a List of Tables.
%\tablespagefalse 

\maketocloflot

\prefacesectiontoc{Abstract}
%\input{abstract} % This file (abstract.tex) contains the text
                 % for an abstract or type text here.
    Porous materials are important across a wide range of applications, including water filtration, catalyst supports, 
    battery electrodes, and bioengineered materials. To accommodate a broader range of pore sizes and enable scalable 
    fabrication with minimal waste, bottom-up synthesis techniques have gained increasing attention. Emulsion templating 
    leverages thermodynamically or kinetically arrested structures formed by phase-separating fluids. Among these, bicontinuous 
    interfacially jammed emulsion gels (bijels) are particularly promising due to their tortuous and co-continuous microstructures.

    Bijels form via thermally induced spinodal decomposition of partially miscible fluid mixtures in the presence of neutrally 
    wetting particles. As the fluid domains coarsen, particles adsorb onto the interface until the interfacial area matches the 
    total cross-sectional area of the particles, resulting in a jammed monolayer that locks the microstructure in place. Traditional 
    bijel synthesis via thermal phase separation is not readily scalable. However, newer techniques such as Solvent Transfer Induced 
    Phase Separation (STrIPS) have been developed to address this limitation, utilizing the removal of solvent from a bijel casting
    mixture to induce phase separation. However, the microstructure obtained is coupled to the casting mixture composition, 
    and the flow rate during STrIPS.
    
    One promising approach to modulate bijel microstructure is through stimulus-responsive systems. Magnetic stimuli, in particular, 
    offer a non-invasive, controllable, and targeted mechanism. Earlier studies using spherical particles under magnetic fields showed 
    limited microstructural changes. More recent work, however, has revealed that ellipsoidal particles can respond to magnetic actuation 
    by tilting, leading to the formation of particle chains or rings at interfaces. This phenomenon opens the door to controllable 
    modifications in bijel microstructures using magnetic fields.
    
    This dissertation investigates whether constant magnetic fields can be used to control the microstructure of bijels stabilized by 
    magnetically responsive ellipsoidal particles. To explore this, we employ a multicomponent Lattice Boltzmann Method coupled with a 
    molecular dynamics representation of rigid particles and a dipole-based magnetic field model. The system simulates a water-2,6-lutidine 
    mixture stabilized by micron-sized nickel-coated polystyrene particles.
    
    We first examine the ability of magnetic fields to influence bijel formation by applying a constant field during spinodal decomposition 
    of fluid mixtures containing disc-like, spherical, and rod-like particles. While no significant change in domain size is observed for 
    spherical particles, bijels stabilized by discs and rods show slight increases in domain size and become anisotropic, as measured by 
    tortuosity and directional domain size.Further analysis reveals that the coarsening rates become direction-specific for ellipsoidal particles, 
    suggesting that jamming occurs anisotropically due to particle alignment with the magnetic field. This alignment, governed by the 
    orientation of the magnetic moment relative to the particle's long axis, also affects how particles arrange themselves at the fluid 
    interface. Particle reorientation affects the curvature of the interface as particles with smaller cross sectional area are less disruptive
    to the saddle like interface shape that minimal surfaces possess. Topological examinations of the interface surface, 
    demonstrating that the number of interconnected channels in the system decreased over time and that the average channel sizes obtained follow
    the same trend as that of the domain size.
    
    Having established that magnetic fields can influence microstructure during synthesis, we then assess whether similar control can be exerted 
    after bijel formation. This has implications for applications like crossflow reactors and filtration systems, where reversible control over 
    permeability and flow resistance is desirable. By incrementally increasing and decreasing the magnetic field on a bijel stabilized with 
    rod-like particles, we observe that the microstructure can indeed be modified post-synthesis. Domain size increases nonlinearly with 
    applied field strength and remains altered even after the field is reduced, indicating a saturation threshold and a history-dependent 
    response.
    
    To probe this further, we evaluate field-driven coarsening in bijels stabilized with rod and disc-like particles. Upon field application, 
    particles reorient, leading to anisotropic domains similar to those formed during synthesis under a field. The temporal evolution of 
    microstructure involves increased particle ordering, interface alignment, and rearrangement, demonstrating a complex interplay of factors. 
    Notably, the extent of domain size change is negatively correlated with the initial ordering of particles, suggesting a memory effect 
    rooted in the starting microstructure.
    
    Given that many magnetically responsive materials exhibit field-dependent rheological behavior (e.g., shear thickening in ferrofluids), 
    we investigate whether bijels exhibit similar effects. Prior studies have shown that shear can induce domain coarsening and particle 
    ejection in bijels. Additionally, emulsions stabilized by ellipsoidal particles show viscosity decreases with increased ordering due 
    to weakened particle-fluid interactions. In this work, we explore how magnetic fields and initial microstructure affect the shear 
    response of bijels stabilized by ellipsoidal particles.As particle ordering increases, the viscosity and shear-thinning behavior decrease for 
    bijels stabilized by disc- and rod-like particles. Interestingly, the yield stress decreases with increased order for discs, but increases for 
    rods due to differences in local particle arrangement and frictional contacts. Under shear, domain coarsening occurs due to shear induced particle 
    tilting out of the interface.
    
    In conclusion, this dissertation demonstrates that magnetic fields offer a viable and tunable method for controlling bijel microstructures both 
    during and after synthesis, and it provides a detailed characterization of the constant shear response of magnetically responsive bijels. These findings 
    lay the groundwork for developing adaptive, magnetically responsive porous materials through both conventional synthesis techniques and Solvent 
    Transfer Induced Phase Separation (STrIPS), enabling scalable and application-specific design.

\end{front}