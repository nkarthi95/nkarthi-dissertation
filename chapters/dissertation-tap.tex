% 
% This is the Title and Approval Page file (dissertation-tap.tex) for
% a dissertation.
%
% The order of the commands below is very important.
% You may choose to add or eliminate a \prefacesection 
% in the front material but the order should remain 
% the same especially \maketocloflot followed by 
% \prefacesectiontoc{Abstract}

% Title and author are also used for PDF file properties
% No special character or commands can be used for the PDF definition; 
% use the [options] paramater to specify a different title or author 
% to remove special characters or commands like \\ for example.
\title[First Line of Title Second Line of Title]{Emulsions Stabilized by Magnetic Ellipsoidal Particles: A Lattice Boltzmann Study}
\author{Nikhil Karthikeyan}
\type{dissertation}
\degree{Doctor of Philosophy}
\majorfieldtrue\majorfield{Materials Science and Engineering}
\degreedate{Spring 2025}
% Optional PDF properties
\keywords{Keyord,Keyword,Keyword}
\subject{Subject}

\maketitlepage % Generates Title Page

\begin{approvalpage}
\chair{Joshua Zide, PhD}{Chair of the Department of Materials Science and Engineering}
\dean{Pamela M. Norris, PhD}{Dean of the College of Engineering}
\end{approvalpage}

\begin{signedpage} % Up to 4 signatures
\profmember{Ulf D. Schiller, PhD}
\member{Eric Furst, PhD}
\member{Arthi Jayaraman, PhD}
\member{Darrin Pochan, PhD}
\end{signedpage}

% For additional signatures beyond 4, uncomment and use
% \begin{signedpagecont}
% \member{Xxxx Xxxx, Highest Degree}
% \member{Xxxx Xxxx, Highest Degree}
% \end{signedpagecont}

\begin{front} % Starts front material (Roman style page numbers)

\prefacesection{Acknowledgements}
%\input{acknowl} % This file (acknowl.tex) contains the text
                % for the acknowledgments or type text here.

In the development of this dissertation, I extend my deepest gratitude to Dr Ulf Schiller whose support, motivation and 
opportunities have been indispensible during my PhD. I believe his methodical problem solving strategies, depth and 
breadth of knowledge have been invaluable sources of inspiration for me as well as 
motivation to strive towards those ideals. I would also like to thank members of my committee, 
from the University of Delaware and Clemson University for their mentorship 
and motivation that pushed me to increase my breadth of knowledge and to think outside the box.

Listing everyone I am grateful for who have provided me non-academic support would be too long for this page. 
I am grateful to my family, with whom I may not be the best at 
discussing my future plans, but they have been an indispensible source of support during my time away from home. 
I am thankful to my lab members especially for adding brief moments of levity during my PhD and giving me the 
opportunity to be a part of their own journey in graduate school. Thank you to everyone who has been with me 
through thick and thin, and to many more celebrations of our achievements. 

% Table of Contents is always created, but you
% may set \tablespagefalse and \figurespagefalse 
% if you don't want these generated automatically
% (i.e. List of Tables and List of Figures).
% These are set to true by default (i.e. \tablespagetrue,
% \figurespagetrue).

% Uncomment if you do not want a List of Figures.
%\figurespagefalse

% Uncomment if you do not want a List of Tables.
%\tablespagefalse 

\maketocloflot

\prefacesectiontoc{Abstract}
%\input{abstract} % This file (abstract.tex) contains the text
                 % for an abstract or type text here.
    Porous materials are important in a wide range of applications, including water filtration, 
    catalyst supports, battery electrodes, and bioengineered materials. To accommodate a broader range of pore 
    sizes and enable scalable fabrication with minimal waste, bottom-up synthesis techniques have gained 
    increasing attention. Among these, emulsion templating based on bicontinuous interfacially jammed 
    emulsion gels (bijels)—has emerged as a promising approach due to its ability to produce complex, 
    interconnected porous structures.

    Despite these advantages, achieving microstructural control independent of the casting mixture remains a 
    key challenge, particularly in continuous bijel fabrication methods such as Solvent Transfer Induced Phase 
    Separation (STrIPS). Additionally, in situ control strategies for tailoring properties such 
    as separation efficiency can be useful additions to control material performance. 
    Stimuli-responsive approaches have been identified as a potential route to 
    overcome these constraints. This study focuses on the magnetic stimuli-response of bijels stabilized by 
    magnetically responsive ellipsoidal microparticles, examining both the resulting microstructural 
    modifications and their rheological implications.

    It is shown that the application of a magnetic field during bijel formation induces microstructural 
    changes in systems stabilized by ellipsoidal particles—effects not observed in bijels stabilized by 
    spherical particles. The applied field causes particle reorientation along its direction, leading to 
    directional jamming and the development of anisotropic microstructures.

    Post-synthesis, bijels stabilized by ellipsoidal particles exhibit tunable microstructure. This responsiveness 
    is governed by the degree of particle order present prior to field application, as well as the extent of 
    reordering induced by the field. The system exhibits state-dependent behavior, indicating that the timing of 
    magnetic field application significantly influences the magnitude of the microstructural response characterized.

    Rheological characterization reveals that bijels retain their shear-thinning behavior under applied magnetic 
    fields. However, the degree of particle ordering is found to influence both the evolution of the microstructure 
    under shear and the corresponding rheological response, highlighting the interplay between field-induced 
    structuring and mechanical performance.
\end{front}