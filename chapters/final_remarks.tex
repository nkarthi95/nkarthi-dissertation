\section{Conclusion}

Porous materials, owing to their high surface-area-to-volume ratio, have gained significant attention in a wide range of 
applications including catalysis, battery electrodes, and pharmaceuticals. Their functionality across diverse sectors has 
driven a growing interest in fabrication techniques capable of accessing and tuning pore structures across multiple length 
scales. Emulsion templating has emerged as a particularly versatile synthesis route, offering access to a broad spectrum of 
microstructures, the potential for stimuli-responsive behavior, and compatibility with a wide variety of material systems. 
Among the structures that can be generated through emulsion templating, the bicontinuous interfacially jammed emulsion gel 
(bijel) stands out for its unique interconnected morphology and potential for functional material design.

Traditional bijel fabrication relies on thermally induced phase separation (TIPS), a method that, while effective, does not 
support continuous production. More recently developed techniques such as Solvent Transfer Induced Phase Separation (STrIPS) 
and Vapor Induced Phase Separation (VIPS) enable continuous fabrication and offer greater control over microstructure. However, 
these methods typically require modifications to the initial emulsion formulation, coupling the casting mixture to the resulting 
morphology. Decoupling these parameters would offer significantly greater flexibility in bijel design, allowing for independent 
tuning of material properties and processing conditions.

External stimuli have been previously employed to modulate the microstructure of particle-stabilized emulsions, with magnetic 
fields offering particular promise due to their directionality and the relatively low energy required to induce a response. Prior 
studies have shown that bijels stabilized with spherical particles exhibit limited structural response to magnetic fields. In 
contrast, anisotropic particles such as ellipsoids have demonstrated the capacity to reorient and tilt out of fluid interfaces 
under magnetic influence. However, the effect of magnetic fields on bijels stabilized by ellipsoidal particles remains unexplored.

This study addresses that gap through a series of hybrid simulations using a coupled Lattice Boltzmann-Molecular Dynamics framework. 
We investigated how magnetic fields influence bijel formation and structure when stabilized by ellipsoidal particles, focusing on 
three key aspects: the impact of magnetic field strength on microstructure compared to systems stabilized by spherical particles; 
the structural response of these bijels under post-formation magnetic fields and how this response depends on the initial ordering 
of the interfacial particle monolayer; and the rheological behavior of such systems under constant shear, particularly in relation 
to prior magnetic field exposure and pre-existing interfacial particle alignment.

We first demonstrated that ellipsoidal particles can be used to modify the microstructure of bijels during synthesis of bijels.
We also explored how varying magnetic flux densities and particle aspect ratios affect structural outcomes. 
In the absence of a magnetic field, ellipsoidal particles aligned with the interface over time due to capillary interactions, contributing to 
anisotropic coarsening rates and finer domain features compared to spherical particles. Upon introducing magnetic fields, particles experienced 
additional torques that promoted alignment along the field direction during formation. This field-induced reorientation improved interfacial 
packing efficiency and introduced directional cessation of coarsening, leading to bijels with domain anisotropy.

Structural characterization of bijels stabilized by ellipsoids revealed morphological detail across particle shapes and magnetic field strengths. 
Six-fold bond orientational order emerged within the interfacial particle layers, but true crystalline order remained suppressed due to the 
inherent curvature of the interface. Ellipsoidal particles increased the Gaussian curvature of the interface compared 
to spheres, while increasing field strength tended to reduce this curvature by promoting particle alignment due to reductions in the interface
deformation of each particle. This reduction in curvature indicated a more hyperbolic interfacial character.

Topological analysis showed that the number of handles, or interconnected channels, decreased over time as coarsening progressed, following 
a power-law relationship inversely proportional to domain size. After the onset of jamming, this decay slowed considerably. Notably, 
channel size distributions (CSDs) revealed a wide range of channel dimensions but did not show systematic variation with 
field strength, suggesting that magnetic field effects manifest more prominently in local structure and domain anisotropy. The 
average channel size derived from the CSD aligned well with conventional domain measurements, reinforcing the results obtained throughout
the course of this work.

Upon applying magnetic fields post formation, we observed structural rearrangements in bijels depending on the initial particle 
configuration. Systems with disordered interfacial layers underwent domain coarsening and anisotropic growth due to field-induced unjamming 
and subsequent re-jamming. Specifically, oblate particles induced perpendicular domain elongation of up to 60\%, while prolate particles led 
to axial coarsening of up to 40\%. These shape-dependent responses were driven by particle rotation and interface realignment under magnetic 
torque, and were quantified using average interface angles and Steinhardt $Q_6$ bond orientational order parameters. Importantly, once the 
magnetic field was removed, the bijel structure largely remained in its field-aligned state. Only minor particle relaxation occurred, 
indicating that capillary forces alone were insufficient to reverse the structural transformations, revealing a that the particle monolayer
was kinetically arrested.

The rheological behavior of these systems was equally sensitive to particle shape, magnetic field history, and interfacial order. All 
bijels exhibited shear-thinning behavior, which is characteristic of non-Newtonian fluids. However, yield stress and flow 
response varied significantly across systems. Prolate particle-stabilized bijels exhibited higher yield stress and more consistent flow 
behavior with increasing magnetic alignment, attributed to improved interfacial adhesion and stress transfer. Conversely, oblate-stabilized 
systems showed a decrease in yield stress when subjected to the same field history. This contrast was explained by the tendency of oblate 
particles to tilt away from the interface under magnetic alignment, making the interfacial layer more susceptible to shear-induced buckling.

Steinhardt $Q_6$ values and interface angle analyses further supported these interpretations. Higher $Q_6$ values in aligned prolate systems 
indicated robust interfacial ordering and enhanced structural resistance under flow, while oblate systems with elevated interface angles 
demonstrated weaker anchoring and a greater likelihood of particle dislodgment. The impact of particle geometry on interfacial mechanics 
and flow resistance highlights the importance of shape-selective design for applications requiring tunable rheological properties, such as 
drug delivery, 3D printing, or adaptive templating.

This work provides several insights that advance the current understanding of field-responsive emulsions and structurally 
tunable soft materials. By integrating ellipsoidal particles with magnetic field stimuli, we demonstrate a new axis of control over 
bijel formation and function. Conceptually, our results refine the understanding of interfacial jamming in bijels. The identification of
unjamming-rejamming mechanisms, modulated by both field strength and particle ordering, reveals that bijel structures are not passively 
determined by initial phase separation dynamics. Instead, they are dynamically reconfigurable under external fields

While prior studies have focused heavily on global measures such as domain size or volume fraction, our use of implemented bond orientational order, 
curvature metrics, and channel topology creates a more nuanced toolkit for evaluating local and global bijel structure. This multiscale 
perspective provides better predictive capacity for tailoring materials with application-specific structural features, such as tortuosity 
for transport limited applications or local ordering for mechanical performance.

The broader impacts of this work span several disciplines. In materials science, our findings inform the design of smart scaffolds and porous 
networks with tunable permeability or rheology. For instance, magnetic field-tunable tortuosity or anisotropy could be leveraged in filtration 
membranes that dynamically adjust flow pathways or in tissue scaffolds where curvature-sensitive cell adhesion is critical. In energy systems, 
the ability to control interfacial curvature and connectivity may influence electrochemical transport and reaction kinetics in battery electrodes 
or catalytic substrates. Additionally, the demonstrated control over yield stress and flow behavior through particle shape and magnetic history 
opens new pathways for engineering responsive soft materials in applications such as injectable therapeutics, adaptive 3D printing inks, or 
magnetorheological fluids.

\section{Future work}

Building on the observed coupling between interfacial particle dynamics and bijel microstructure, we propose extending 
this work by investigating bijels stabilized with non-rigid, cohesive, or attractive particles. Unlike rigid ellipsoids, 
soft and cohesive particles are known to exhibit distinct glass transition behaviors and rheological responses compared 
to hard-sphere systems \cite{weeks_introduction_2017, torquato_jammed_2010}. In bijel architectures, attractive interactions have been 
proposed as a strategy to form “armored” interfaces, potentially enhancing mechanical integrity and enabling higher flow rates in catalytic 
applications \cite{boakye-ansah_controlling_2020}. 
Similarly, soft particles such as nanogels and hydrogels have attracted interest in biomedical contexts, offering tunable responsiveness 
to environmental stimuli like temperature and pH—critical features for controlled drug delivery. Incorporating such particles 
could alter jamming dynamics, interfacial rearrangement, and field-induced responsiveness. These effects may be 
explored computationally using Lattice Boltzmann Methods integrated with immersed boundary method based frameworks to account for
particle deformation or attractive particle interactions to account for electrostatic cohesion \cite{silva_lattice_2024}. This direction opens opportunities 
to design bijel systems with enhanced adaptability and function, tailored for application-specific demands.

The current simulations, while able to capture structural evolution under external fields, do not fully resolve the complex dynamical 
landscape of the interfacial particle monolayer. In particular, they lack the resolution to distinguish between fast particle rearrangements 
and slow collective dynamics near the glass transition, which are known to govern the jamming and unjamming behavior of dense 
colloidal systems \cite{weeks_introduction_2017, torquato_jammed_2010}. 
As such, the interplay between interfacial ordering, particle-scale dynamics, and existing order can be characterized more convincingly by 
investigating the dynamics of the particle monolayer more convincingly by accounting for thermal fluctuations of the fluid. This would involve 
implementation of a fluctuating Lattice Boltzmann Method which would facilitate identification of the glassy dynamics in the system, 
including the onset of arrest and structural aging characteristic of colloidal glasses \cite{weeks_introduction_2017, torquato_jammed_2010}.

Another avenue for future investigation involves the use of gradient or rotating magnetic fields in place of constant fields. This
would impart a force on the particles in addition to the torque that magnetic particles experience under 
constant fields \cite{spatafora-salazar_hierarchical_2021, martinezpedrero_collective_2020}.
Prior work has demonstrated that spatial gradients in particle volume fraction can generate domain size gradients in bijels, enabling spatial 
control over porosity and connectivity \cite{french_bicontinuous_2022}. Analogously, applying a magnetic field gradient may induce vertical 
gradients in nematic order, influencing local particle packing and yielding position-dependent pore structures across the bijel. This could be 
particularly useful for creating functionally graded materials with hierarchical transport properties.

\section{Acknowledgments}

The author acknowledges Dr. Ulf Schiller and the members of the Schiller and Kuksenok groups for the discussions on 
the characterization and computational techniques used in this work. This work is supported by the US National Science 
Foundation under award numbers DMR-1944942 and OIA-2131996. Any opinions, findings, conclusions, or recommendations 
expressed in this material are those of the author(s) and do not necessarily reflect those of the National Science 
Foundation.  

This research was supported in part through the use of DARWIN computing system: DARWIN - A Resource for Computational 
and Data-intensive Research at the University of Delaware and in the Delaware Region, Rudolf Eigenmann, Benjamin E. 
Bagozzi, Arthi Jayaraman, William Totten, and Cathy H. Wu, University of Delaware, 2021.

This work used Delta at the University of Illinois Urbana Champaign through allocation PHY220131 from the Advanced 
Cyberinfrastructure Coordination Ecosystem: Services $\&$ Support (ACCESS) program, which is supported by National 
Science Foundation grants 2138259, 2138286, 2138307, 2137603, and 2138296. 

Clemson University is acknowledged for generous allotment of compute time on Palmetto cluster. This research used the 
Delta advanced computing and data resource which is supported by the National Science Foundation (award OAC 2005572) 
and the State of Illinois. Delta is a joint effort of the University of Illinois Urbana-Champaign and its National 
Center for Supercomputing Applications. 