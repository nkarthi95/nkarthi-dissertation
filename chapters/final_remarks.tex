\section{Conclusion}

Owing to their high surface area to volume ratio, porous materials are seeing a surge in popularity in applications
such as catalysis, battery electrodes and pharmaceuticals. Given their wide variety of applications, identifying 
fabrication techniques that allow access to the various pore length scales is of interest. One such synthesis technique
is emulsion templating which offers a wide variety of accessible microstructures, addition of stimuli response and 
large number of possible systems that can be fabricated. A microstructure that can be fabricated from emulsion templates
is the bicontinuous interfacially jammed emulsion gel (bijel). 

Bijels are normally fabricated using Thermally Induced Phase Separation (TIPS). However, TIPS does not allow for continuous
fabrication of bijels. More recent fabrication techniques such as Solvent Transfer Induced Phase Separation (STrIPS) and 
Vapor Induced Phase Separation (VIPS) allow for continuous fabrication and access to various microstructures. Both techniques
require modifications to the initial emulsion mixture to facilitate microstructure adjustments, which affect the final rheological
properties of the system. Decoupling the microstructure and the casting mixture of the bijel would allow for greater flexibility
in the synthesis of the material.

Stimuli response has been used before to modify the microstructure of particle stabilized emulsions. Magnetic fields offer
targeted material response and low applied field strengths necessary for response. Past work investigating the effect of magnetic
stimuli on bijels stabilized with spherical particles yielded little microstructure change. However, anisotropic particles have
also been used to stabilize bijels. Anisotropic particles at interfaces under magnetic fields have been shown to tilt out of 
interfaces. These effects have yet to be captured in bijels stabilized by ellipsoidal particles under magnetic fields.
This study addresses this knowledge gap using a hybrid Molecular-Dynamics multicomponent method Lattice Boltzmann Method.
We split this work into three aims; Aim 1 addressed the microstructure obtained when applying a magnetic field of various strengths 
onto bijels stabilized by ellipsoidal particles, compared to bijels stabilized wih spherical particles. Aim 2 addressed the 
structural response of bijels stabilized by ellipsoidal particles and analyzes the effect of initial order of the particle monolayer
on the structural response observed. Aim 3 addressed the constant shear response of bijels stabilized by ellipsoidal particles
with pre-existing particle order and under magnetic fields.

In Aim 1, bijels stabilized with spherical particles do not respond to the application of a magnetic field. However, bijels stabilized
with ellipsoidal particles have an increase in the average domain size by 3 \%. The microstructure also becomes anisotropic, 
characterized as a change in the directional tortuosity and distribution of channel widths as a function of the distance from the 
interface. The response characterized was particle morphology specific due to the orientation of the particle magnetic moment with 
respect to the direction of the long axis of the particle. This causes the particles to arrange differently on the interface. The 
microstructure anisotropy is caused by direction specific jamming of the particle monolayer originating from particle ordering to 
the magnetic field. The direction specific jamming originates from the orientation of the particles to the magnetic field. The direction
specific jamming also reduces the local curvature of the system as the interface location moves with the particle.

To investigate the role of the particle monolayer in greater detail, we characterize the orientational order to the magnetic field,
angle to the interface and local particle ordering at the interface. The orientational order changes as a function of the applied field 
strength and has particle morphology specific time evolution. When investigating the average interfacial angle of the particles to the 
interface, it is shown how the capillary interactions with the interface differ, leading to the time evolution differences observed.
When analyzing the ordering of particles on the interface, disc-like particles see lowered local ordering as the magnetic field strength is 
increased while rod-like particles see greater local ordering. These properties are attributed to how those particle morphologies prefer 
to orient themselves at interfaces to one another, with disc like particles preferring to stack while rod like particles order side to side 
or end to end.

In Aim 2, bijels stabilized by ellipsoidal particles demonstrate responses to an applied magnetic field. When applying a magnetic field
onto a bijel template simulated with no magnetic fields, we see an average microstructure change of up to 5 \%. This microstructure change
arises from the magnetic field driven re-orientation of particles at the interface. Particle reorientation creates microstructure anisotropy
arising from the alignment of particles to the magnetic field. When analyzing the particle monolayer, we characterize the alignment of particles
to the field direction, average interface angle and the local ordering of the particles. We demonstrate that the average interface angle increases
as the particles reorient to the field before the capillary forces of the interfaces causes the interface to move, until jamming of the monolayer takes
place. The local particle order during this process decreases for oblate particles and increases for prolate particles. We characterize that the local 
order of the particles controls the timescales of response.

When investigating the effect of pre-existing order of the particle monolayer, we show that the structural response of the bijel is dependent upon the
difference between the field applied to make the bijel and the applied field when the field is applied in the same direction. We also characterize that the
local particle monolayer is dependent upon the pre-existing order of the bijel. When switching off the magnetic field onto bijels made with a magnetic field,
we see that the microstructure remains resilient to the removal of the magnetic field. We characterize that while there are differences in the particle monolayer,
these differences manifest as when domain coarsening begins. When characterizing the hysteresis curve of the bijels stabilized with prolate particles,
we see the effect of this.

In Aim 3 we investigate the rheological response of bijels stabilized with ellipsoidal particles under constant shear and show that the application of magnetic
fields changes the shear thinning properties of bijels. 

\section{Future work}

This work utilized 2 ellipsoidal particle geometries chosen for comparisons to previous literature using this 
particle geometry. Particles based on cellulose nanocrystals or graphene nanoplates are now in use to fabricate
particle stabilized emulsions.These particles have also been shown to have capillary bridging and particle stacking,
not seen in the particles used in this work. These particles in bulk have been shown to have some intrinsic ordering
that can be predicted using onsager theory. An investigation into how using rods or plate like particles would be 
instructive in identifying if onsager theory can be used in bijel formation and its link to bijel microstructure. 
\textcolor{blue}{https://doi.org/10.1039/D1SM00367D}

Colloidal systems made with cohesive and soft particles have been shown in the literature to have different glass 
transition points and rheological behavior from their hard sphere counterparts. In bijels, cohesive particles in 
particular have been suggested as a means to create "armored" bijels to improve their performance in catalytic materials, 
allowing higher flow rates to be used. Investigations into soft particles are of interest in biomedical engineering 
applications, with newly developed nanogels and hydrogels being suggested as drug carriers or vectors for stimuli 
response through temperature or pH. LBM methods that implement the immersed boundary method can be used to model soft
particles with a DLVO potential used to model electrostatics between particles. \textcolor{blue}{https://doi.org/10.1039/D3SM01648J}

Another avenue of exploration would be the use of gradient or rotating magnetic fields in place of the constant magnetic 
fields used here. One study on bijel microstructure showed that a gradient in the particle volume fraction can be used to 
create a gradient in the eventual domain size. A gradient field may be able to generate a gradient in the nematic order 
parameter, affecting the particle packing of the bijel at different heights and varying the pore size as a function of 
height in the field gradient direction. In a ferrofluid or solution of magnetic colloids, a rotating field can assemble 
particles into chains or rings. In the context of bijel structural response, this can be used to tune bijel microstructure 
as this process can control the unjamming and rejamming of the particle monolayer, allowing for greater control over the 
resulting bijel microstructure than a constant field would have. In these simulations, the frequency of rotation would 
likely need to be tuned based upon how quickly particles respond to field in the bijel.


% Extensions to this work can be accomplished by investigating the effect of applying a magnetic field on the bijel 
% while under shear. Ferrofluid models that predict bingham plastic like flows, $\frac{\eta}{eta_{f}} = 1 + \frac{Mn^{*}(\phi_p)}{Mn}$, 
% have been developed and defined using the Mason number, $Mn = \frac{8\eta_{f} \dot{\gamma}}{\mu_{0} \mu_{f} \beta^{2} H_0^2}$. 
% \textcolor{blue}{https://doi.org/10.1122/1.4935850, https://linkinghub.elsevier.com/retrieve/pii/S1359029405000385} 

\section{Acknowledgments}

The author acknowledges Dr. Ulf Schiller and the members of the Schiller and Kuksenok groups for the discussions on 
the characterization and computational techniques used in this work. This work is supported by the US National Science 
Foundation under award numbers DMR-1944942 and OIA-2131996. Any opinions, findings, conclusions, or recommendations 
expressed in this material are those of the author(s) and do not necessarily reflect those of the National Science 
Foundation.  

Clemson University is acknowledged for generous allotment of compute time on Palmetto cluster. This research used the 
Delta advanced computing and data resource which is supported by the National Science Foundation (award OAC 2005572) 
and the State of Illinois. Delta is a joint effort of the University of Illinois Urbana-Champaign and its National 
Center for Supercomputing Applications. 

This work used Delta at the University of Illinois Urbana Champaign through allocation PHY220131 from the Advanced 
Cyberinfrastructure Coordination Ecosystem: Services $\&$ Support (ACCESS) program, which is supported by National 
Science Foundation grants 2138259, 2138286, 2138307, 2137603, and 2138296. 
