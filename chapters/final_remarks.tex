\section{Conclusion}

\section{Future work}

This work utilized 2 ellipsoidal particle geometries which were not as anisotropic as rods or plates that some particles 
have naturally, such as cellulose nanocrystals and graphene nanoplates. These particles have also been shown to naturally 
have nematic ordering at higher volume fractions that correspond to the volume fractions present at bijel interfaces. An 
investigation into how using rods or plate like particles would be instructive in identifying how onsager theory can be 
linked to bijel microstructure. \textcolor{blue}{https://doi.org/10.1039/D1SM00367D}

Colloidal systems made with cohesive and soft particles have been shown in the literature to have different glass 
transition points and rheological behavior from their hard sphere counterparts. In bijels, cohesive particles in 
particular have been suggested as a means to create "armored" bijels to improve their performance in catalytic materials, 
allowing higher flow rates to be used. Investigations into soft particles are of interest in biomedical engineering 
applications, with newly developed nanogels and hydrogels being suggested as drug carriers or vectors for stimuli 
response through temperature or pH.

Another avenue of exploration would be the use of gradient or rotating magnetic fields in place of the constant magnetic 
fields used here. One study on bijel microstructure showed that a gradient in the particle volume fraction can be used to 
create a gradient in the eventual domain size. A gradient field may be able to generate a gradient in the nematic order 
parameter, affecting the particle packing of the bijel at different heights and varying the pore size as a function of 
height in the field gradient direction. In a ferrofluid or solution of magnetic colloids, a rotating field can assemble 
particles into chains or rings. In the context of bijel structural response, this can be used to tune bijel microstructure 
as this process can control the unjamming and rejamming of the particle monolayer, allowing for greater control over the 
resulting bijel microstructure than a constant field would have. In these simulations, the frequency of rotation would 
likely need to be tuned based upon how quickly particles respond to field in the bijel.


% Extensions to this work can be accomplished by investigating the effect of applying a magnetic field on the bijel 
% while under shear. Ferrofluid models that predict bingham plastic like flows, $\frac{\eta}{eta_{f}} = 1 + \frac{Mn^{*}(\phi_p)}{Mn}$, 
% have been developed and defined using the Mason number, $Mn = \frac{8\eta_{f} \dot{\gamma}}{\mu_{0} \mu_{f} \beta^{2} H_0^2}$. 
% \textcolor{blue}{https://doi.org/10.1122/1.4935850, https://linkinghub.elsevier.com/retrieve/pii/S1359029405000385} 

\section{Acknowledgments}

The author acknowledges Dr. Ulf Schiller and the members of the Schiller and Kuksenok groups for the discussions on 
the characterization and computational techniques used in this work. This work is supported by the US National Science 
Foundation under award numbers DMR-1944942 and OIA-2131996. Any opinions, findings, conclusions, or recommendations 
expressed in this material are those of the author(s) and do not necessarily reflect those of the National Science 
Foundation.  

Clemson University is acknowledged for generous allotment of compute time on Palmetto cluster. This research used the 
Delta advanced computing and data resource which is supported by the National Science Foundation (award OAC 2005572) 
and the State of Illinois. Delta is a joint effort of the University of Illinois Urbana-Champaign and its National 
Center for Supercomputing Applications. 

This work used Delta at the University of Illinois Urbana Champaign through allocation PHY220131 from the Advanced 
Cyberinfrastructure Coordination Ecosystem: Services $\&$ Support (ACCESS) program, which is supported by National 
Science Foundation grants 2138259, 2138286, 2138307, 2137603, and 2138296. 
