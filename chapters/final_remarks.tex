\section{Conclusion}

Porous materials, owing to their high surface-area-to-volume ratio, have gained significant attention in a wide range of 
applications including catalysis, battery electrodes, and pharmaceuticals. Their functionality across diverse sectors has 
driven a growing interest in fabrication techniques capable of accessing and tuning pore structures across multiple length 
scales. Emulsion templating has emerged as a particularly versatile synthesis route, offering access to a broad spectrum of 
microstructures, the potential for stimuli-responsive behavior, and compatibility with a wide variety of material systems. 
Among the structures that can be generated through emulsion templating, the bicontinuous interfacially jammed emulsion gel 
(bijel) stands out for its unique interconnected morphology and potential for functional material design.

Traditional bijel fabrication relies on thermally induced phase separation (TIPS), a method that, while effective, does not 
support continuous production. More recently developed techniques—such as Solvent Transfer Induced Phase Separation (STrIPS) 
and Vapor Induced Phase Separation (VIPS)—enable continuous fabrication and offer greater control over microstructure. However, 
these methods typically require modifications to the initial emulsion formulation, coupling the casting mixture to the resulting 
morphology. Decoupling these parameters would offer significantly greater flexibility in bijel design, allowing for independent 
tuning of material properties and processing conditions.

External stimuli have been previously employed to modulate the microstructure of particle-stabilized emulsions, with magnetic 
fields offering particular promise due to their directionality and the relatively low energy required to induce a response. Prior 
studies have shown that bijels stabilized with spherical particles exhibit limited structural response to magnetic fields. In 
contrast, anisotropic particles such as ellipsoids have demonstrated the capacity to reorient and tilt out of fluid interfaces 
under magnetic influence. However, the effect of magnetic fields on bijels stabilized by ellipsoidal particles remains unexplored.

This study addresses that gap through a series of hybrid simulations using a coupled Molecular Dynamics-Lattice Boltzmann framework. 
We investigated how magnetic fields influence bijel formation and structure when stabilized by ellipsoidal particles, focusing on 
three key aspects: the impact of magnetic field strength on microstructure compared to systems stabilized by spherical particles; 
the structural response of these bijels under post-formation magnetic fields and how this response depends on the initial ordering 
of the interfacial particle monolayer; and the rheological behavior of such systems under constant shear, particularly in relation 
to prior magnetic field exposure and pre-existing interfacial particle alignment.

We first demonstrated that ellipsoidal particles alter the formation dynamics of bijels when subjected to external magnetic 
fields during phase separation. We also explored how varying magnetic flux densities and particle aspect ratios affect structural outcomes. 
In the absence of a magnetic field, ellipsoidal particles 
aligned with the interface over time due to capillary interactions, contributing to slower coarsening and finer domain features compared 
to spherical particles. Upon introducing magnetic fields, particles experienced additional torques that promoted alignment along the field 
direction during formation. This field-induced reorientation improved interfacial packing efficiency and introduced directional cessation
of coarsening, leading to bijels with domain anisotropy. Importantly, these effects were achieved without preventing the formation of a 
stable, bicontinuous structure.

Structural characterization of bijels stabilized by ellipsoids revealed morphological detail across particle shapes and magnetic field strengths. 
Six-fold bond orientational order emerged within the interfacial particle layers, but true crystalline order remained suppressed due to the 
inherent curvature of the interface. Ellipsoidal particles increased the Gaussian curvature of the interface compared 
to spheres, while increasing field strength tended to reduce this curvature by promoting particle alignment due to reductions in the interface
deformation of each particle. This reduction in curvature indicated a more hyperbolic interfacial character.

Topological analysis showed that the number of handles, or interconnected channels, decreased over time as coarsening progressed, following 
a power-law relationship inversely proportional to domain size. After the onset of jamming, this decay slowed considerably. Notably, 
channel size distributions (CSDs) revealed a wide range of channel dimensions but did not show systematic variation with 
field strength, suggesting that magnetic field effects manifest more prominently in local structure and domain anisotropy. Still, the 
average channel size derived from the CSD aligned well with conventional domain measurements, reinforcing the results obtained throughout
the course of this work.

Upon applying magnetic fields post formation, we observed significant structural rearrangements in bijels depending on the initial particle 
configuration. Systems with disordered interfacial layers underwent domain coarsening and anisotropic growth due to field-induced unjamming 
and subsequent re-jamming. Specifically, oblate particles induced perpendicular domain elongation of up to 60\%, while prolate particles led 
to axial coarsening of up to 40\%. These shape-dependent responses were driven by particle rotation and interface realignment under magnetic 
torque, and were quantified using average interface angles and Steinhardt $Q_6$ bond orientational order parameters. Importantly, once the 
magnetic field was removed, the bijel structure largely remained in its field-aligned state. Only minor particle relaxation occurred, 
indicating that capillary forces alone were insufficient to reverse the structural transformations, revealing a hysteresis-like behavior 
in the response to magnetic stimuli.

The degree of structural reorganization was closely linked to the initial state of the particle monolayer. We identified two primary 
response modes. The first is where initially disordered layers unjammed and reorganized under magnetic fields before settling into a new 
jammed configuration, and another where partially ordered particles reoriented to align with the field, modifying domain structure 
without full unjamming. Systems with high initial order showed minimal structural change, suggesting that the dynamic response of bijels 
can be pre-programmed via the interfacial order established during formation.

The rheological behavior of these systems was equally sensitive to particle shape, magnetic field history, and interfacial order. All 
bijels exhibited shear-thinning behavior, which is characteristic of non-Newtonian fluids. However, yield stress and flow 
response varied significantly across systems. Prolate particle-stabilized bijels exhibited higher yield stress and more consistent flow 
behavior with increasing magnetic alignment, attributed to improved interfacial adhesion and stress transfer. Conversely, oblate-stabilized 
systems showed a decrease in yield stress when subjected to the same field history. This contrast was explained by the tendency of oblate 
particles to tilt away from the interface under magnetic alignment, making the interfacial layer more susceptible to shear-induced buckling.

Steinhardt $Q_6$ values and interface angle analyses further supported these interpretations. Higher $Q_6$ values in aligned prolate systems 
indicated robust interfacial ordering and enhanced structural resistance under flow, while oblate systems with elevated interface angles 
demonstrated weaker anchoring and a greater likelihood of particle dislodgment. The impact of particle geometry on interfacial mechanics 
and flow resistance highlights the importance of shape-selective design for applications requiring tunable rheological properties, such as 
drug delivery, 3D printing, or adaptive templating.

This work provides several key insights that significantly advance the current understanding of field-responsive emulsions and structurally 
tunable soft materials. By integrating ellipsoidal particle design with magnetic field stimuli, we demonstrate a new axis of control over 
bijel formation and function. 
Conceptually, our results refine the understanding of interfacial jamming in bijels. The identification of distinct 
unjamming-rejamming mechanisms, modulated by both field strength and particle ordering, reveals that bijel structures are not passively 
determined by initial phase separation dynamics. Instead, they are dynamically reconfigurable under external fields, with the extent of 
reconfiguration governed by the memory of prior structural states. This introduces a paradigm in which bijels can be treated as 
non-equilibrium materials with tunable hysteresis, offering a controllable blend of stability and adaptability.

Our work also implements structural characterization for describing bijel microstructures. While prior studies have focused heavily on global 
measures such as domain size or volume fraction, our use of bond orientational order, curvature metrics, and channel topology creates a more 
nuanced toolkit for evaluating local and global bijel structure. This multiscale perspective provides better predictive capacity for 
tailoring materials with application-specific structural features, such as tortuosity for mass transport or local ordering for mechanical 
performance.

The broader impacts of this work span several disciplines. In materials science, our findings inform the design of smart scaffolds and 
porous networks with tunable permeability or rheology. For instance, magnetic field-tunable tortuosity or anisotropy could 
be leveraged in filtration membranes that dynamically adjust flow pathways or in tissue scaffolds where curvature-sensitive cell adhesion 
is critical. In energy systems, the ability to control interfacial curvature and connectivity may influence electrochemical transport and 
reaction kinetics in battery electrodes or catalytic substrates.

The rheological tunability demonstrated here has direct implications for soft robotics, adaptive damping materials, and controlled release 
systems. The distinction between oblate and prolate behavior under shear and field stresses provides a foundation for selecting particle 
geometries based on the desired flow performance of the end-use material. Furthermore, the discovery that certain configurations can resist 
flow better than others may lead to new materials with programmable viscoelasticity—materials that stiffen or soften in response to field inputs.

\section{Future work}

This work utilized 2 ellipsoidal particle geometries chosen for comparisons to previous literature using this 
particle geometry. Particles based on cellulose nanocrystals or graphene nanoplates are now in use to fabricate
particle stabilized emulsions. These particles have also been shown to have capillary bridging and particle stacking,
not seen in the particles used in this work. These particles in bulk have been shown to have intrinsic ordering
that can be predicted using onsager theory. An investigation into how using rods or plate like particles would be 
instructive in identifying if onsager theory can be used in bijel formation and its link to bijel microstructure. 
\cite{tan_2d_2021}

Colloidal systems made with cohesive and soft particles have been shown in the literature to have different glass 
transition points and rheological behavior from their hard sphere counterparts. In bijels, attractive particles in 
particular have been suggested as a means to create "armored" bijels to improve their performance in catalytic materials, 
allowing higher flow rates to be used. Investigations into soft particles are of interest in biomedical
applications, with newly developed nanogels and hydrogels being suggested as drug carriers or vectors for stimuli 
response through temperature or pH. LBM methods that implement the immersed boundary method can be used to model soft
particles with a DLVO potential used to model electrostatics between particles. \cite{silva_lattice_2024}

Another avenue of exploration would be the use of gradient or rotating magnetic fields in place of the constant magnetic 
fields used here. One study on bijel microstructure showed that a gradient in the particle volume fraction can be used to 
create a gradient in the eventual domain size. A gradient field may be able to generate nematic order 
parameter gradients, affecting the particle packing of the bijel at different heights and varying the pore size as a function of 
height in the field gradient direction. In a ferrofluid or solution of magnetic colloids, a rotating field can assemble 
particles into chains or rings. In the context of bijel structural response, this can be used to tune bijel microstructure 
as this process can control the unjamming and rejamming of the particle monolayer, allowing for greater control over the 
resulting bijel microstructure than a constant field would have. In these simulations, the frequency of rotation would 
likely need to be tuned based upon how quickly particles respond to field in the bijel.


% Extensions to this work can be accomplished by investigating the effect of applying a magnetic field on the bijel 
% while under shear. Ferrofluid models that predict bingham plastic like flows, $\frac{\eta}{eta_{f}} = 1 + \frac{Mn^{*}(\phi_p)}{Mn}$, 
% have been developed and defined using the Mason number, $Mn = \frac{8\eta_{f} \dot{\gamma}}{\mu_{0} \mu_{f} \beta^{2} H_0^2}$. 
% \textcolor{blue}{https://doi.org/10.1122/1.4935850, https://linkinghub.elsevier.com/retrieve/pii/S1359029405000385} 

\section{Acknowledgments}

The author acknowledges Dr. Ulf Schiller and the members of the Schiller and Kuksenok groups for the discussions on 
the characterization and computational techniques used in this work. This work is supported by the US National Science 
Foundation under award numbers DMR-1944942 and OIA-2131996. Any opinions, findings, conclusions, or recommendations 
expressed in this material are those of the author(s) and do not necessarily reflect those of the National Science 
Foundation.  

Clemson University is acknowledged for generous allotment of compute time on Palmetto cluster. This research used the 
Delta advanced computing and data resource which is supported by the National Science Foundation (award OAC 2005572) 
and the State of Illinois. Delta is a joint effort of the University of Illinois Urbana-Champaign and its National 
Center for Supercomputing Applications. 

This work used Delta at the University of Illinois Urbana Champaign through allocation PHY220131 from the Advanced 
Cyberinfrastructure Coordination Ecosystem: Services $\&$ Support (ACCESS) program, which is supported by National 
Science Foundation grants 2138259, 2138286, 2138307, 2137603, and 2138296. 

This research was supported in part through the use of DARWIN computing system: DARWIN - A Resource for Computational 
and Data-intensive Research at the University of Delaware and in the Delaware Region, Rudolf Eigenmann, Benjamin E. 
Bagozzi, Arthi Jayaraman, William Totten, and Cathy H. Wu, University of Delaware, 2021